
\documentclass[]{cv}
\usepackage{fontawesome}
\usepackage{multicol}

%------------
\usepackage{enumitem}
\renewenvironment{entrylist}{%
  \begin{itemize}[leftmargin=1in]%[leftmargin=*,align=left,itemindent=-\dimexpr\labelwidth+\labelindent+\labelsep\relax]
}{%
  \end{itemize}
}
\renewcommand{\bfseries}{\headingfont\color{headercolor}}
\renewcommand{\entry}[4]{%
  \item[#1]
    \textbf{#2}%
    \hfill%
    {\footnotesize\addfontfeature{Color=lightgray} #3}\\%
    #4\vspace{\parsep}%
  }
  
\makecvfooter
  {\today}
  {Alejandro González Pérez}
  {CV}
%-------

\begin{document}
\header{Alejandro }{González}
       {\emph{"Computer science is no more about computers than astronomy is about telescopes."} -Edsger Dijkstra}


% In the aside, each new line forces a line break
\begin{aside}
	\section{info}
	Spanish
	November 1990
	~~~
    \section{contact}
        \href{https://algope.com}{\faGlobe\ algope.com}
        \href{https://t.me/algope}{\faPhone\ +4528270147}
        \href{mailto:hi@algope.com}{\faEnvelope\ hi@algope.com}
        \href{https://goo.gl/maps/pjW26KgQ3oeQMkN19}{\faStreetView\ København, DK}
        ~~~
    \section{social}
        \href{https://www.linkedin.com/in/algope}{\faLinkedin\ algope} 
        \href{http://github.com/algope}{\faGithub\ algope}
        \href{http://twitter.com/algope_}{\faTwitter\ algope\_}
        \href{https://www.instagram.com/algope_}{\faInstagram\ algope\_}
        ~~~
    \section{dev}
        R
        GO
        AWS
        Java
        C/C++
        Python
        NodeJS
        JavaScript
        ~~~
    \section{interests}
        Politics
        Equality
        Big Data
        Open Data
        Civic Hacking
        Human Rights
        Open Government
        Artificial Intelligence
        ~~~
     \section{languages}
        English: C2
        Swedish: A1
        Catalan: Native
     	Spanish: Native
      	~~~
\end{aside}

\section{experience}
\begin{entrylist}
\entry
    {Jan20 - current}
    {Senior Software Engineer}
    {TELIA COMPANY, København (Denmark) \\
    \href{https://www.teliacompany.com}{teliacompany.com}}
    {\textbf{Skills}: Java, AWS, CI/CD, Microservices}
\entry
    {Jun18 - Dec19}
    {Product Research \& Development}
    {TELIA COMPANY, Stockholm (Sweden) \\
    \href{https://www.teliacompany.com}{teliacompany.com}}
    {\textbf{Skills}: Python, ML, NLP, DL}
\entry
    {Jan18 - Jun18}
    {Master Thesis - Machine Learning}
    {TELIA COMPANY, Stockholm (Sweden) \\
    \href{https://www.teliacompany.com}{teliacompany.com}}
    {\textbf{Skills}: Python, ML, NLP, DL}
\entry
    {Mar17 - Aug17}
    {Intern - Data Engineer}
    {3ANTS, Madrid (Spain) \\
    \href{http://3ants.com}{3ants.com}}
    {\textbf{Skills}: Python, MongoDB, Scraping, Spark, ML, DL}
\entry
    {Oct16 - Jul17}
    {Guest Lecturer}
    {UNIVERSITAT POLITÈCNICA de VALÈNCIA (UPV), València (Spain) \\
    \href{http://www.upv.es}{upv.es}}
    {\textbf{Skills}: Teaching, Communication}
  \entry
    {Sep14 Jan15}
    {Intern - Front-end Developer}
    {EHUMANLIFE, Inc., València (Spain)\\
    \href{http://ehumanlife.com}{ehumanlife.com}}
    {\textbf{Skills}: JavaScript, JQuery, HTML5, CSS3}
    
  \entry
    {Feb14 Sep14}
    {Intern - Software Engineer}
    {EVERIS, València (Spain) \\
    \href{http://everis.com}{everis.com}}
    {\textbf{Skills}: Team work, Testing, Java EE, PostgreSQL, Oracle DB, Pentaho}
  \entry
    {Jul13 Jul14}
    {Co-founder and Vice-chair}
    {ACM UPV Chapter, València (Spain)\\
    \href{http://acmupv.webs.upv.es}{acmupv.webs.upv.es}}
    {\textbf{Skills}: Leadership, Oratory, Group Management}
\end{entrylist}


\section{education}
\begin{entrylist}
 \entry
        {2019-current}
        {Doctor of Philosophy}
        {Polytechnic University of Valencia}
        {Artificial Intelligence in Communication and Media Studies}
    \entry
        {2017-2018}
        {Teknologie Mastersexamen ICT Innovation}
        {Kungliga Tekniska högskolan}
        {Specialisation in Data Science, EIT Digital programme}
    \entry
        {2016-2017}
        {MSc. Computer Science}
        {Polytechnic University of Madrid}
        {Specialisation in Data Science, EIT Digital programme}
    \entry
        {2010-2016}
        {B.Sc. Computer Science }
        {Polytechnic University of Valencia}
        {Specialisation in Information Technology}
\end{entrylist}



\section{volunteering}
test
\begin{entrylist}	
  \entry
    {Sep12 Jul14}
    {Mentor for exchange students}
    {Polytechnic University of Valencia}
    {}
  \entry
    {Feb12 Feb13}
    {University Relations Coordinator}
    {Polytechnic University of Valencia}
    {}
\end{entrylist}

\section{projects}

\begin{entrylist}
  \entry
    {2015}
    {civicBOT}
    {\href{https://github.com/algope/civicbot}{github.com/algope/civicbot}}
    {\emph{A chatbot for monitoring the performance of political parties}}
  \entry
    {2015}
    {amigaBOT}
    {\href{https://github.com/algope/amigaBOT}{github.com/algope/amigaBOT}}
    {\emph{A chatbot to allow victims of domestic violence report a risky situation}}
  \entry
    {2016}
    {votingBOT}
    {\href{https://github.com/algope/votingBOT}{github.com/algope/votingBOT}}
    {\emph{A chatbot that eases voting processes}
    \\\textbf{Winner} - Used in a real referendum held by the city of Quart de Poblet, Valencia (Spain)}
  \entry
    {2017}
    {OVE}
    {\href{https://github.com/openvademecum}{github.com/openvademecum}}
    {\emph{An open source drug information API}}
  \entry
    {2017}
    {BärlinerBOT}
    {\href{http://bit.ly/barliner}{bit.ly/barliner}}
    {\emph{A chatbot connecting citizens with government services}
    \\\textbf{Winner} - EIT Digital Citizen Participation and City Governance}
   \entry
    {2019}
    {MediaData}
    {\href{https://mediadata.webs.upv.es/en/}{https://mediadata.webs.upv.es/en/}}
    {\emph{A space for the development and innovation of technology for content analysis of mass media.}}
   
    
\end{entrylist}

\section{certifications}
\begin{entrylist}
  \entry
    {Sep15}
    {M101JS: MongoDB for Node.js Developers}
    {MongoDB, Inc.}
    {}
  \entry
    {Mar16}
    {Inferential Statistics}
    {Udacity}
    {}
  \entry
    {Apr16}
    {TOEFL iBT}
    {ETS}
    {99}
\end{entrylist}

\end{document}
